\section{Methodology}\label{sec:methodology}
We build on the technique proposed by Halderman (ref to the initial paper that
proposed this).
\textbf{ discuss basic idea of the approach and add a summary of shortcomings
of that approach; what practicalities does it ignore?} 

\subsection{Dealing with Multiple Paths}
Because load-balancers can affect our measurements and conclusions, we fix the
paths our probes take in the following fashion (in line with the literature):
\begin{itemize}
    \item For ICMP:
    \item For DNS:
    \item For TCP/HTTP:
\end{itemize}
\textbf{Do our probes across different protocols take wildly different paths?
We need some validation on this, and may need a strategy to identify and filter
the cases that do not.}

\subsection{Reference (ICMP) Path Validation}
\begin{itemize}
    \item So and so papers have shown that routers and servers can block ICMP
        probes.  We first establish the rate of such blocking (manifested by
        incomplete paths).
     \item Experiment: For a sample of top Alexa websites (10K), check
         reachability (path completeness and path inflation) to their
         authoritative name-servers and webservers over ICMP.  
     \item \textit{Path inflation:} Mention how we discover it: we started with
         TTL of 25 and were getting a high percentage of incomplete paths.
         Increasing the TTL to 64, increases the coverage of complete paths,
         but inflates it.
     \item Show empirical results and build an argument for discarding
         incomplete and inflated paths.
     \item Confirm that this happens from all vantage points.
\end{itemize}

\subsection{Path Validation Data Collection}
We decided to use Alexa Top 1M list, and randomly selected top 1000 domains from 100000 top domains. We did this to ensure that our current sample has sufficient variability with respect to where the servers are hosted, and who these servers belong to. For example, there a lot of www.google.** domains in top Alexa domains.


We firstly resolve those 1000 domains using a custom resolver code[Link code's repo]. We used, this custom script and not the local resolver to make sure that the authoritative server of that domain replies to our IP rather than our local resolver's IP which move out of our domain of control. So this step was taken to ensure repeatability. This resolver code takes in a list of of domains and returns a file which has data in the following format. \\ 
\begin{center}
   domain, cname, authserverIP, webserverIP \\    
\end{center}
We then divide the list into two sets, one set contains the list of domain and its corresponding authserverIP and the second set contains the list of domain and its corresponding webserverIP. We currently have rented four servers located at following locations: \\
1. USA\\
2. Russia\\
3. Turkey\\
4. China\\

Unfortunately we were unable to run any measurement beyond resolution on Chinese machine primarily because its a CentOS machine and we had to recompile a lot of scripts for that particular environment. With that said, we were able to resolve and run measurements from those three vantage points and collect data. 

We ran a couple of preliminary measurements and found out that a lot of ICMP probes were not reaching the server, so our primary hypothesis was that most of the path were getting inflated because of IDS. So proposed way to measure this was to send ICMP packets with a higher TTL value so that IDS cases get eliminated and we have a solid set of paths onto which we can base our methodology. 


So in this experiment, as mentioned earlier we would run two different set of traceroutes on different sets of resolved servers. For the Authoritative server we will send ICMP and UDP (DNS packets with query to get current bind version) probes. For Webservers we will send ICMP, TCP and HTTP packets. We set the intial TTL value to be default traceroute value which is 25, and then we collected the results and analyzed them and then ran the same experiment with TTL value of 64.

We made a couple of assumptions in our analysis, first of all we assumed was in path inflation code. We assumed that if there are more than two non responding consecutive IPs in the traceroute we assumed that the path is inflated because of IDS. 

Secondly, we are assuming that if there is no reply for fifteen consecutive hops then the server will not respond for the max ttl value either. Which essentially implies that IDS is completely blocking our access.

Before we go into the analyzing the results, we will first enlist the base numbers to give weight to the percentages. \\

In Russia there were a total of 805 unique webserver, and 869 unique authservers.
In USA there were a total of 793 unique webserver, and 871 unique authservers.
In Turkey there were a total of 788 unique webserver, and 867 unique authservers.


On a very high level, the results show a difference in percentage of ICMP probes that reach the server. On average around 85 percent of ICMP probes with TTL 25, and 98 percent ICMP probes when TTL value is 64, reach authoritative servers. Similarly, for in case of webservers we saw roughly equal reach-ability of about 75 percent for both 25 and 64 TTL. 

There was a significantly higher path length inflation in ICMP paths to authservers compared to webservers (average of 25 percent vs 10 percent). 

Broadly speaking, the TTL configuration of 64 with 15 gap limit would be ideal way to probe both webservers and authoritative servers, however there is a much higher incidence of ICMP blocking around webservers compared to authservers. In worst case we would be dropping around 20 percent of ICMP probes (if we dont fix the path inflation) for webservers and around 35 percent in case of authservers.

\begin{table*}
\small

    \begin{center}

    \begin{tabular}{|l|r|r|r|r|r|r|r|r|} \hline
        \multirow{4}*{Vantage Point} &
        \multicolumn{4}{c|}{Authoritative Nameservers Reachability} 
         &
        \multicolumn{4}{c|}{Webservers Reachability}
         \\ \cline{2-9}
        & 
        \multicolumn{2}{c|}{\% Complete Paths} 
        &
        \multicolumn{2}{c|}{\% Possible Inflated Paths}
        &
        \multicolumn{2}{c|}{\% Complete Paths} 
        &
        \multicolumn{2}{c|}{\% Possible Inflated Paths}
        \\ \cline{2-9}



        & \multirow{2}*{TTL=25} &
          \multirow{2}*{TTL=64} &
          \multirow{2}*{TTL=25} &
          \multirow{2}*{TTL=64} &
          \multirow{2}*{TTL=25} &
          \multirow{2}*{TTL=64} &
          \multirow{2}*{TTL=25} &
          \multirow{2}*{TTL=64} 

                        \\ 
                        
        &&&&&&&& \\  \hline

        Russia
        & 87.2 & 98.3 & 23.2 & 23.5 & 75.1 & 75.8 & 10.7 & 11.1
             \\ \hline

        Turkey
        & 84.3 & 99.5 & 33.2 & 36.5 & 75.6 & 78.0 & 9.9 & 11.9 
             \\ \hline

        USA
        & 84.7 & 98.6 & 22.8 & 25.1 & 73.7 & 76.5 & 7.1 & 9.5
             \\ \hline
        
        China
        & NA & NA & NA & NA & NA & NA & NA & NA
             \\ \hline

    \end{tabular}

    \caption{ICMP reachability for Y websites. We start with the
        Alexa top 10,000.  Filter down to the set that have the same
        authoritative name server from all vantage points, resulting in n =  X.
        We further filter down to the set that have the same webserver across
    all vantage points, resulting in n = Y.}
    \label{tab:pathvalidation}

    \end{center}
\end{table*}


\begin{table*}
\small

    \begin{center}

    \begin{tabular}{|l|r|r|} \hline
        Number of Vantage Points &
        Authoritative Nameservers 
        (n=) &
        Webservers
        (n=)

        \\ \hline

        None    &    &  \\ \hline
        Only one &   & \\ \hline
        Only two &   &  \\ \hline
        All Three &  &  \\ \hline
       
       
    \end{tabular}

    \caption{ICMP reachability for Y websites using three back-to-back ICMP
        probes (TTL = 64). We start with the
        Alexa top 1,000.  Filter down to the set that have the same
        authoritative name server from all vantage points, resulting in n =  X.
        We further filter down to the set that have the same webserver across
    all vantage points, resulting in n = Y.}
    \label{tab:ICMPreachabilityacrossvantages}

    \end{center}
\end{table*}
