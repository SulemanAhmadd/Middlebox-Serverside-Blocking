\section{Methodology}\label{sec:methodology}
We build on the technique proposed by Halderman (ref to the initial paper that
proposed this).
\textbf{ discuss basic idea of the approach and add a summary of shortcomings
of that approach; what practicalities does it ignore?} 

\subsection{Dealing with Multiple Paths}
Because load-balancers can affect our measurements and conclusions, we fix the
paths our probes take in the following fashion (in line with the literature):
\begin{itemize}
    \item For ICMP:
    \item For DNS:
    \item For TCP/HTTP:
\end{itemize}
\textbf{Do our probes across different protocols take wildly different paths?
We need some validation on this, and may need a strategy to identify and filter
the cases that do not.}

\subsection{Reference (ICMP) Path Validation}
\begin{itemize}
    \item So and so papers have shown that routers and servers can block ICMP
        probes.  We first establish the rate of such blocking (manifested by
        incomplete paths).
     \item Experiment: For a sample of top Alexa websites (10K), check
         reachability (path completeness and path inflation) to their
         authoritative name-servers and webservers over ICMP.  
     \item \textit{Path inflation:} Mention how we discover it: we started with
         TTL of 25 and were getting a high percentage of incomplete paths.
         Increasing the TTL to 64, increases the coverage of complete paths,
         but inflates it.
     \item Show empirical results and build an argument for discarding
         incomplete and inflated paths.
     \item Confirm that this happens from all vantage points.
\end{itemize}


\begin{table*}
\small

    \begin{center}

    \begin{tabular}{|l|r|r|r|r|r|r|r|r|} \hline
        \multirow{4}*{Vantage Point} &
        \multicolumn{4}{c|}{Authoritative Nameservers Reachability} 
         &
        \multicolumn{4}{c|}{Webservers Reachability}
         \\ \cline{2-9}
        & 
        \multicolumn{2}{c|}{\% Complete Paths} 
        &
        \multicolumn{2}{c|}{\% Possible Inflated Paths}
        &
        \multicolumn{2}{c|}{\% Complete Paths} 
        &
        \multicolumn{2}{c|}{\% Possible Inflated Paths}
        \\ \cline{2-9}



        & \multirow{2}*{TTL=25} &
          \multirow{2}*{TTL=64} &
          \multirow{2}*{TTL=25} &
          \multirow{2}*{TTL=64} &
          \multirow{2}*{TTL=25} &
          \multirow{2}*{TTL=64} &
          \multirow{2}*{TTL=25} &
          \multirow{2}*{TTL=64} 

                        \\ 
                        
        &&&&&&&& \\  \hline

        Russia
        &&&&&&&&
             \\ \hline

        China
        &&&&&&&&
             \\ \hline

        Pakistan
        &&&&&&&&
             \\ \hline
        
        USA
        &&&&&&&&
             \\ \hline

    \end{tabular}

    \caption{ICMP reachability for Y websites. We start with the
        Alexa top 10,000.  Filter down to the set that have the same
        authoritative name server from all vantage points, resulting in n =  X.
        We further filter down to the set that have the same webserver across
    all vantage points, resulting in n = Y.}
    \label{tab:pathvalidation}

    \end{center}
\end{table*}
